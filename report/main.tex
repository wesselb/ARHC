%!TEX program = xelatex

\documentclass[letterpaper,11pt]{extarticle}
% Euler math font
\usepackage{eulervm}
\usepackage[T1]{fontenc}

% General
\usepackage[no-math]{fontspec} % Specify fonts
\usepackage[english]{babel}    % Naming of figures and such
\usepackage[
    style=ieee,
    sortlocale=en_GB
    ]{biblatex}                % Bibliography styling
\usepackage{import}            % Importing subdocuments
\usepackage{standalone}        % Compilable subdocuments
\usepackage{hyperref}          % Clickable references
\usepackage{csquotes}          % Required for biblatex

% Page layout
\usepackage[
    a4paper,
    bottom=1.2in,
    top=1.2in,
    left=0.8in,
    right=0.8in,
    headheight=14pt
    ]{geometry}               % Margins
\usepackage{fancyhdr}         % Header
\usepackage{lastpage}         % Page numbers in footer
\usepackage{setspace}         % Line spacing
\usepackage{ragged2e}         % Better line endings
\usepackage[parfill]{parskip} % New line instead of indentation
\usepackage{titlesec}         % Size of sections

% Utility
\usepackage[table]{xcolor}                                       % Colors
\usepackage[separate-uncertainty=true, per-mode=symbol]{siunitx} % Display units
\usepackage{enumitem}                                            % Better enumeration
\usepackage{ifthen}                                              % If-statements
\usepackage{lipsum}                                              % Lorem ipsum
\usepackage{algpseudocode}
\usepackage{algorithm}

% Figures
\usepackage{pgfplots}                   % Plots
\pgfplotsset{compat=newest}             % Pgfplots compatibility
\usepackage{tikz}                       % Tikz figures
\usepackage{float}                      % Floating of figures
\usepackage{ltablex}                    % Merge of tabularx and longtable (use tabularx if stuff breaks)
\usepackage{multirow}                   % Cells spanning multiple rows
\usepackage{graphicx}                   % Graphics
\usepackage{caption}                    % Subfigures and captions
\captionsetup{justification=centering}
\usepackage{subcaption}                 % Subfigures and captions
\usepackage{booktabs}                   % Toprule etc
\usepackage{wrapfig}                    % Wrapping around figures if necessary

% Math
\usepackage{amsmath} % Math commands
\usepackage{amsthm}         % Math environments
\usepackage{amssymb}        % Math symbols

% Cleveref
\usepackage[noabbrev, capitalize]{cleveref} % Automatic referencing

% Custom math
% Vectors
\renewcommand{\vec}[1]{\boldsymbol{#1}} % Vector
\newcommand{\vecu}[1]{\hat{\vec{#1}}}   % Unit vector
\newcommand{\mat}[1]{\vec{#1}}          % Matrix

% Set notations
\newcommand{\Z}{\mathbb{Z}}   % Integers
\newcommand{\R}{\mathbb{R}}   % Real numbers
\renewcommand{\C}{\mathbb{C}} % Complex numbers
\newcommand{\N}{\mathbb{N}}   % Natural numbers

% Symbols
\newcommand{\es}{\varnothing}   % Empty set
\newcommand{\e}{\varepsilon}    % Epsilon
\newcommand{\sub}{\subseteq}    % Subset
\newcommand{\imp}{\Rightarrow}  % Implication
\renewcommand{\d}{\partial}     % Partial
\renewcommand{\th}{\theta}      % Theta

% Operators
\renewcommand{\Re}{\operatorname{Re}}
\renewcommand{\Im}{\operatorname{Im}}
\newcommand{\Cov}{\operatorname{Cov}}
\newcommand{\Var}{\operatorname{Var}}
\newcommand{\argmax}[1]{\underset{#1}{\operatorname{argmax}}~}
\newcommand{\argmin}[1]{\underset{#1}{\operatorname{argmin}}~}
\newcommand{\tr}{\operatorname{tr}}
\newcommand{\col}{\operatorname{col}}
\newcommand{\proj}{\operatorname{proj}}
\renewcommand{\perp}{\operatorname{perp}} % \bot for perp symbol
\newcommand{\sign}{\operatorname{sign}}
\newcommand{\rank}{\operatorname{rank}}

% Small superscript in math
\newcommand{\sus}[1]{^{\text{\tiny $#1$}}}
\newcommand{\supsmall}[1]{\sus{#1}}        % Explicit command

% Subscripts without spacing
\renewcommand{\ss}[1]{_\mathit{#1}}

% Straight 'd' in integral
\newcommand{\id}[1]{\, \mathrm{d} #1}

% Amsmath environements
\newtheorem{theorem}{Theorem}
\newtheorem{lemma}{Lemma}
\newtheorem{definition}{Definition}

% Fix interference of parfill option of parskip package with amsmath environments
\begingroup
    \makeatletter
    \@for\theoremstyle:=definition,remark,plain\do{%
        \expandafter\g@addto@macro\csname th@\theoremstyle\endcsname{%
            \setlength\thm@preskip\parskip
            \setlength\thm@postskip{0pt}
            \addtolength\thm@preskip\parskip
            }%
        }
\endgroup


% Hyperlink
\hypersetup{
    colorlinks,
    citecolor = black,
    filecolor = black,
    linkcolor = black,
    urlcolor  = black
}

% Left-justified text in tabularx environment
\newcolumntype{L}{>{\RaggedRight\arraybackslash}X}

% Fonts
\setsansfont{Avenir Next LT Pro}
\setromanfont{Adobe Caslon Pro}
\setmonofont[Scale=0.85]{Source Code Pro}
\newcommand{\code}[1]{\texttt{#1}} % Code

% Style enumerate
\setlist[enumerate]{label=(\arabic*)}
\setlist[itemize]{label=\textbullet}

% Shortcuts to colors
\newcommand{\red}[1]{{\color{red} #1}}
\newcommand{\blue}[1]{{\color{blue} #1}}
\newcommand{\green}[1]{{\color{green} #1}}

% Line spacing
\onehalfspacing

% Page layout
\pagestyle{fancy}
\lhead{\DocTitle\ifthenelse{\equal{\DocCourse}{}}{}{, \DocCourse}}
\chead{}
\rhead{\DocAuthor}
\lfoot{}
\cfoot{}
\rfoot{Page~\thepage~of~\pageref{LastPage}}
\renewcommand{\headrulewidth}{0pt}
\renewcommand{\footrulewidth}{0pt}

\newcommand{\DocCourse}{}
\newcommand{\DocTitle}{}
\newcommand{\DocAuthor}{}

% Title and sections
\titleformat*{\section}{\Large\bfseries}
\titlespacing*{\section}{0pt}{\baselineskip}{0.5\baselineskip}
\titleformat*{\subsection}{\large\bfseries}
\titlespacing*{\subsection}{0pt}{\baselineskip}{0.5\baselineskip}
\renewcommand{\title}{\vspace*{-1.4em}{\LARGE\bfseries\DocTitle}\vspace*{0.5\baselineskip}}

% Sections in margins
\newcommand{\sectionsInMargins}{
    \titleformat{\section}[leftmargin]
        {\Large\bfseries\filleft}
        {\thesection) }{0pt}{}
        \titlespacing{\section}
        {4pc}{1.5ex plus .1ex minus .2ex}{1pc}
}
\newcommand{\subsectionsInMargins}{
    \titleformat{\subsection}[leftmargin]
        {\large\bfseries\filleft}
        {\thesubsection) }{0pt}{}
        \titlespacing{\subsection}
        {4pc}{1.5ex plus .1ex minus .2ex}{1pc}
}

% Run in on sections
\newcommand{\runInSection}{
    \titleformat{\section}[runin]
        {\normalfont\Large\bfseries}{\thesection}{1em}{}
}

\newcommand{\runInSubsection}{
    \titleformat{\subsection}[runin]
        {\normalfont\Large\bfseries}{\thesubsection}{1em}{}
}

% Reduced space below figures
% \setlength{\belowcaptionskip}{-1em}

% Easy section creation
\newcommand{\newSection}[1]{\section{} \label{sec:#1}}

% Listings
\usepackage{listings}
\newcommand\listingStyle{\lstset{
    otherkeywords    = {self},
    basicstyle       = \fontsize{9}{8}\selectfont\ttfamily,
    showstringspaces = false,
    breaklines       = true,
    postbreak        = \raisebox{0ex}[0ex][0ex]{\ensuremath{\color{red}\hookrightarrow\space}},
    tabsize          = 2,
    escapeinside     = {(*}{*)},
    mathescape
}}

\lstnewenvironment{listing}[1][]
  {\vspace{0.25em}\listingStyle\lstset{#1}}
  {\vspace{-0.75em}}


% Table footnote
\newcommand{\tableFootnote}[1]{
    \\[0.5em]
    {\centering \footnotesize #1}
    \vspace{-0.5em}
}

% Scientific e notation
\newcommand{\sce}{\text{\sc{e}}}

% Framed listing
\newcommand{\framedListings}{
    \lstset{
         frame     = single,
         framerule = 0.075em,
         rulecolor = \color{black}
    }
}

% Floating listings
\DeclareCaptionFormat{cont}{#1~(cont.)#2#3 }

\newcounter{tmp}

\newcommand\savecount{\setcounter{tmp}{\value{lstlisting}}}
\newcommand\contcap{%
\setcounter{lstlisting}{\value{tmp}}
\addtocounter{lstlisting}{-1}
\captionsetup[lstlisting]{format=cont}}

% Expandable input
\makeatletter
\newcommand*\expandableInput[1]{\@@input#1 }
\makeatother

\addbibresource{bibliography.bib}

\renewcommand{\DocTitle}{Coursework 2}
\renewcommand{\DocCourse}{MLSALT6}
\renewcommand{\DocAuthor}{James Requeima (jrr41), Wessel Bruinsma (wpb23)}

\usepackage{setspace}

\newcommand{\outputPath}{output/}

\begin{document}
\title
\singlespacing

\begingroup

In this document we describe the design of our Adaptive Run-length Huffman Compressor (ARHC).

\section{Motivation}

Since the Huffman code is obliged to use one character per symbol, using it on the symbol set $\{0,1\}$ for the task of compressing $N=10,000$ bits, each having probability $f = 0.01$ of being a 1, will be extremely wasteful. Since the $0$, symbol is highly probable, we transmit an entire bit when virtually no information is conveyed. If we are observing the file bit-by-bit, the number of bits that pass before a one is observed does contain a lot of information. As an approach, we treat each of these zero-counts, or run-lengths, as a symbol for the Huffman algorithm. So that our encoding and decoding algorithm is well defined, we will pick a maximum run-length which will be the largest number of zeros to denote as a symbol.

Now we need to determine the maximum run length. Let $L$ be the expected codelength and $H$ be the entropy of the distribution over the symbols in our symbol set. Then the expected number of wasted bits per symbol is  $L - H$. If we plot the ratio $L/H$ we see that this ratio is pretty close to one at (approximately) multiples of $69$.

%plot

The expected codeword length of the Huffman code is equal to the entropy when the distribution over the symbols in our symbol set is dyadic. Let $r$ be the maximum run length used. The all zero symbol will have probability $(0.99)^r$ and so in order to have  $L/H$ very close to one we must have that $(0.99)^r \approx 2^{-k} \implies r\approx k\times 69$.

\section{Description}

\section{Analysis}

\section{Conclusion}


Code:
\begin{enumerate}
    \item Refactor
    \item Nice comments/docstrings
    \item Tests/doctests
\end{enumerate}

Graphs:
\begin{enumerate}
    \item Compression performance versus $N$
    \item On-the-fly compression performance
    \item Compression performance for incorrect $p$
    \item Compression performance for different priors
\end{enumerate}


\begin{table}[ht]
    \centering
    \vspace{-0.55cm}
    \setstretch{0.55}
    \centerline{\hspace{.5cm}\begin{subfigure}{.6\linewidth}
        \begin{tabular}{lp{3cm}p{4cm}}
            \toprule
            Input bit & Compressor output & Decompressor output \\ \midrule
            \expandableInput{\outputPath tab_static.tex}
            \bottomrule
        \end{tabular}
        \hspace{1cm}\caption{Static compression scheme}
    \end{subfigure}
    \begin{subfigure}{.6\linewidth}
        \begin{tabular}{lp{3cm}p{4cm}}
            \toprule
            Input bit & Compressor output & Decompressor output \\ \midrule
            \expandableInput{\outputPath tab_adaptive.tex}
            \bottomrule
        \end{tabular}
        \caption{Adaptive compression scheme}
    \end{subfigure}}
    \caption{Output of the compressor and decompressor when the input is fed bit by bit. The input consists of the first 80 bits of the bent-coin benchmark file \code{filep.01.10000NR}. The parameters of the prior are $\alpha_0=0.5$ and $\alpha_1=0.5$. The static compression scheme compresses down to 22 bits, the adaptive compression scheme compresses down to 26 bits and the SCT tells us that we cannot compress further than $\lceil80 H_2(0.01) \rceil = 7$ bits.}
    \label{tab:bits_highlight}
\end{table}


\begin{figure}[ht]
    \centering
    \begin{subfigure}{.49\linewidth}
        \includegraphics[width=\linewidth]{\outputPath comparison1}
        \caption{Compressor matches reality, the parameter of the static compressor scheme is $p=0.1$ and the bent coin has probability $p=0.1$}
    \end{subfigure}
    \begin{subfigure}{.49\linewidth}
        \includegraphics[width=\linewidth]{\outputPath comparison1_mismatch}
        \caption{Compressor does not match reality, the parameter of the static compressor scheme is $p=0.1$ and the bent coin has probability $p=0.01$}
    \end{subfigure}
    \caption{Comparison of the static compression scheme and adaptive compression scheme. The parameters of the prior are $\alpha_0=0.1$ and $\alpha_1=0.1$. The lines and associated areas show respectively the means and 95\% confidence regions determined over 200 runs.}
    \label{fig:comparison_mismatch}
\end{figure}


\begin{figure}[ht]
    \centering
    \includegraphics[width=.95\linewidth]{\outputPath comparison_priors}
    \caption{Comparison of the adaptive compression scheme for different $\alpha_0$ and $\alpha_1$. The lines and associated areas show respectively the means and 95\% confidence regions determined over 50 runs.}
    \label{fig:comparison_priors}
\end{figure}

\endgroup
\printbibliography

\end{document}